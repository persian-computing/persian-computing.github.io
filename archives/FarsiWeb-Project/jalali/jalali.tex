%%
%% This is file `jalali.tex',
%% also known as `ircal.tex',
%% Copyright (C) 2000-2004  Behdad Esfahbod <behdad@farsitex.org>
%%
%% Provides Iranian Calendar
%%
\newif\ifir@leap
\newif\ifir@kabiseh

\def\ir@today{{%
\countdef\ir@i=0
\countdef\ir@y=1
\countdef\ir@m=2
\countdef\ir@d=3
\countdef\ir@temp=4
\countdef\ir@temptwo=5
\countdef\ir@tempthree=6
\countdef\ir@yModHundred=7
\countdef\ir@dn=8
\countdef\ir@sn=9
\countdef\ir@mminusone=10
\ir@y=\year \ir@m=\month \ir@d=\day
%
\ir@temp=\ir@y
\divide\ir@temp by 100%
\multiply\ir@temp by 100%
\ir@yModHundred=\ir@y
\advance\ir@yModHundred by -\ir@temp\relax
%
\ifodd\ir@yModHundred
   \ir@leapfalse
\else
   \ir@temp=\ir@yModHundred
   \divide\ir@temp by 2\relax
   \ifodd\ir@temp\ir@leapfalse
   \else
      \ifnum\ir@yModHundred=0%
         \ir@temp=\ir@y
         \divide\ir@temp by 400\relax
         \multiply\ir@temp by 400\relax
         \ifnum\ir@y=\ir@temp\ir@leaptrue\else\ir@leapfalse\fi
      \else\ir@leaptrue
      \fi
   \fi
\fi
%
\chardef\ir@gregi31\ifir@leap\chardef\ir@gregii29\else\chardef\ir@gregii28\fi
\chardef\ir@gregiii31\chardef\ir@gregiv30%
\chardef\ir@gregv31\chardef\ir@gregvi30%
\chardef\ir@gregvii31\chardef\ir@gregviii31%
\chardef\ir@gregix30\chardef\ir@gregx31%
\chardef\ir@gregxi30\chardef\ir@gregxii31%
%
\ir@temp=\ir@y
\advance\ir@temp by -17\relax
\ir@temptwo=\ir@temp
\divide\ir@temptwo by 33\relax
\multiply\ir@temptwo by 33\relax
\advance\ir@temp by -\ir@temptwo
\ifnum\ir@temp=32\relax\ir@kabisehfalse
\else
   \ir@temptwo=\ir@temp
   \divide\ir@temptwo by 4\relax
   \multiply\ir@temptwo by 4\relax
   \advance\ir@temp by -\ir@temptwo
   \ifnum\ir@temp=\z@\ir@kabisehtrue\else\ir@kabisehfalse\fi
\fi
%
% In fact irani is equal to the Leap years from a fixed year to the last
% year minus the Kabise years from a fixed year to the last year plus a const.
%
\ir@tempthree=\ir@y                 % Number of Leap years
\advance\ir@tempthree by\m@ne
\ir@temp=\ir@tempthree              % T := (MY-1) div 4
\divide\ir@temp by 4\relax
\ir@temptwo=\ir@tempthree           % T := T - ((MY-1) div 100)
\divide\ir@temptwo by 100\relax
\advance\ir@temp by -\ir@temptwo
\ir@temptwo=\ir@tempthree           % T := T + ((MY-1) div 400)
\divide\ir@temptwo by 400\relax
\advance\ir@temp by \ir@temptwo
\advance\ir@tempthree by -611       % Number of Kabise years
\ir@temptwo=\ir@tempthree           % T := T - ((SY+10) div 33) * 8
\divide\ir@temptwo by 33\relax
\multiply\ir@temptwo by 8\relax
\advance\ir@temp by -\ir@temptwo
\ir@temptwo=\ir@tempthree           %
\divide\ir@temptwo by 33\relax
\multiply\ir@temptwo by 33\relax
\advance\ir@tempthree by -\ir@temptwo
\ifnum\ir@tempthree=32\advance\ir@temp by\@ne\fi % if (SY+10) mod 33=32 then Inc(T);
\divide\ir@tempthree by 4\relax     % T := T - ((SY+10) mod 33) div 4
\advance\ir@temp by -\ir@tempthree
\advance\ir@temp by -137            % T := T - 137  Adjust the value
\ir@temptwo=31
\advance\ir@temptwo by-\ir@temp     % now 31 - T is the farsii
\chardef\ir@irani\ir@temptwo
\chardef\ir@iranii30
\ifir@kabiseh\chardef\ir@iraniii30\else\chardef\ir@iraniii29\fi
\chardef\ir@iraniv31\chardef\ir@iranv31%
\chardef\ir@iranvi31\chardef\ir@iranvii31%
\chardef\ir@iranviii31\chardef\ir@iranix31%
\chardef\ir@iranx30\chardef\ir@iranxi30%
\chardef\ir@iranxii30\chardef\ir@iranxiii30%
%
\ir@dn=0\relax
\ir@sn=0\relax
\ir@mminusone=\ir@m
\advance\ir@mminusone by\m@ne\relax
%
\ir@i=0\relax
\ifnum\ir@i<\ir@mminusone
  \loop
  \advance \ir@i by\@ne\relax
  \advance\ir@dn by \csname ir@greg\romannumeral\ir@i\endcsname
  \ifnum\ir@i<\ir@mminusone\repeat
\fi
\advance \ir@dn by\ir@d
%
\ir@i=1\relax
\ir@sn=\ir@irani
\ifnum\ir@sn<\ir@dn
  \loop
  \advance\ir@i by1\relax
  \advance\ir@sn by\csname ir@iran\romannumeral\ir@i\endcsname
  \ifnum\ir@sn<\ir@dn\repeat
\fi
\ifnum\ir@i<4%
   \ir@m=9\advance\ir@m by\ir@i
   \advance\ir@y by-622\relax
\else
   \ir@m=\ir@i \advance\ir@m by-3\relax
   \advance\ir@y by-621\relax
\fi
\advance\ir@sn by -\csname ir@iran\romannumeral\ir@i%
\endcsname
\ifnum\ir@i =\@ne
  \ir@d=\ir@dn \advance\ir@d by30 \advance\ir@d by-\ir@irani
\else
  \ir@d=\ir@dn \advance\ir@d by-\ir@sn
\fi
%
\xdef\ir@@d{\the\ir@d}%
\xdef\ir@@m{\the\ir@m}%
\xdef\ir@@y{\the\ir@y}%
}%
\let\ir@d\ir@@d
\let\ir@m\ir@@m
\let\ir@y\ir@@y
}

\endinput
%%
%% End of `jalali.tex'.
